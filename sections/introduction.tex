\section{Introduction}
\label{intro}
\thispagestyle{empty}

Speech synthesis is not a recent ambition in mankind history. The earliest attempts to synthesize speech are only legends starring Gerbert d'Aurillac (died 1003 A.D.), also known as Pope Sylvester II. The pretended system used by him was a brazen head: a legendary automaton imitating the anatomy of a human head and capable to answer any question. Back in those days, the brazen heads were said to be owned by wizards. Following Pope Sylvester II, some important characters in mankind history  were reputed to have one of these heads, such as Albertus Magnus or Roger Bacon.

During the 18th century, Christian Kratzenstein, a German-born doctor, physicist and engineer working at the Russian Academy of Sciences, was able to built acoustics resonators similar to the human vocal tract. He activated the resonators with vibrating reeds producing the the five long vowels: /a/, /e/, /i/, /o/ and /u/.

Almost at the end of the 18th century, in 1791, Wolfgang von Kempelen presented his Acoustic-Mechanical Speech Machine \cite{vonKempelen}, which was able to produce single sounds and some combinations. During the first half of the 19th century, Charles Wheatstone built his improved and more complicated version of Kempelen's Acoustic-Mechanical Speech Machine, capable of producing vowels, almost all the consonants, sound combinations and even some words.	

In the late 1800's, Alexander Graham Bell also built a speaking machine and did some questionable experiments changing with his hands the vocal tract of his dog and making the dog bark in order to produce speech-like sounds \cite{Schroeder93, LemmettyMSc}.

Before World War II, Bell labs developed the vocoder, which analyzed and extract fundamentals tone and frequency from speech. In the 1950's, the first computer based speech synthesis systems were created and in 1968 the first general English text-to-speech (TTS) system was developed at the Electrotechnical Laboratory, Japan \cite{Klatt87}. From that time on, the main branch of speech synthesis development has being focused on electronic systems, but research conducted on mechanical synthesizers has not been abandoned \cite{mechSynthWeb, mechSynth}.

All the different kind of systems described pursued the same goal: produce natural sounding speech, which is the main goal of speech synthesis. As an extra requirement to this main goal, TTS systems aim to create the speech from arbitrary texts given as inputs, increasing the difficulty. It is easy to assume that a considerably amount of data is needed in order to cover all 

