\section{Vocoders}
\label{vocoders}
The interface with both the natural speech and the synthesized speech is the vocoder. In this section, the fundamentals of the vocoder are presented and a detailed description of the two vocoders compared in this project is given.

\subsection{Vocoder}
\label{vocoders_vocoder}
The human speech is produced by regulating the air from the lungs through the throat, mouth and nose. The airflow from the lungs is modulated at the larynx by the vocal folds, creating the main excitation for voiced speech. The airflow is then filter by the vocal tract, formed by the pharynx and the oral and nasal cavities, acting as an acoustic time-varying filter by adjusting the dimensions and volume of the pharynx and the oral cavity.

A vocoder has to main tasks: analyze speech to extract features and synthesize speech from features, i.e. a vocoder is analysis/synthesis method. Depending on the synthesis system used, the vocoders have to be implemented 